% ============= setup ============= %
% ======== package ======== %
\documentclass[mathserif]{beamer}
\usepackage{xeCJK}
\usepackage{graphicx}
\usepackage{xcolor}
\usepackage{setspace}
\usepackage{newtxmath}

% ======== font ======== %
\setCJKmainfont{Taipei Sans TC Beta}
\setCJKsansfont{Taipei Sans TC Beta}
\AtBeginDocument{%
    \DeclareSymbolFont{pureletters}{OML}{cmm}{m}{it}%
    \SetSymbolFont{pureletters}{bold}{OML}{cmm}{b}{it}%
}
\hypersetup{
    colorlinks=true,
    linkcolor=black,
    urlcolor=blue
}

% ======== theme ======== %
\renewcommand{\baselinestretch}{1.25}
\usetheme{Madrid}
\usecolortheme{crane}
\setbeamertemplate{items}[circle]
\setbeamertemplate{section in toc}{\inserttocsectionnumber.~\inserttocsection}
\AtBeginSection[]{
    \begin{frame}
        \vfill
        \centering
        \begin{beamercolorbox}[sep=8pt,center,shadow=true,rounded=true]{title}
            \usebeamerfont{title}\insertsectionhead\par%
        \end{beamercolorbox}
        \vfill
    \end{frame}
}

% ======== data ======== %
\title{git 簡介}
\author{temmie}
\date{}

% ============= setup ============= %

\begin{document}

\begin{frame}
    \titlepage
\end{frame}

\begin{frame}
    \tableofcontents
\end{frame}

\section{git 初認識}

\begin{frame}
    \frametitle{什麼是 git}
    \begin{itemize}
        \item git 是一種版本控制系統
        \item 可以用來追蹤文件的變化
        \item 可以隨時查看或是回溯文件的歷史版本
    \end{itemize}
\end{frame}

\begin{frame}
    \frametitle{git 的優點}
    \begin{itemize}
        \item git 是一款\textbf{開源的分散式}版本控制系統
        \vspace{0.5cm}
        \item<2-> 分散式:每個人都可以在本地擁有檔案,方便協作
        \item<3-> 高效性:因為只需要處理文件有變化的部份,因此操作速度很快
        \item<4-> 易於使用:git 的指令非常簡單
    \end{itemize}
\end{frame}

\end{document}